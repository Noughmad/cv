\documentclass[a4paper]{article}
\usepackage[utf8]{inputenc}
% \usepackage[slovene]{babel}

\title{\v Zivljenjepis}
\author{Miha \v Can\v cula}
\date{Marec 2015}

\pagenumbering{arabic}
\setcounter{page}{2}

\begin{document}
\maketitle

Miha se je rodil 10. aprila 1989 v Postojni, osnovno šolo pa je začel obiskovati v Ljubljani leta 1996. V osnovni šoli se je udeležil večih tekmovanj iz znanja. Prejel je več državnih nagrad, med njimi prva mesta na državnih tekmovanjih v fiziki, kemiji, logiki in računalništvu. 

Po zaključeni osnovni šoli se je vpisal na Gimnazijo Bežigrad in prejel Zoisovo štipendijo za nadarjene učence. V gimnaziji se je posvetil predvsem matematiki in fiziki. Leta 2007 se je udeležil Mednarodne matematične olimpijade v Vietnamu, kjer je prejel bronasto medaljo. Leta 2008 se je udeležil Mednarodne fizikalne olimpijade, prav tako v Vietnamu, kjer je spet prejel bronasto medaljo. Istega leta se je udeležil tudi Mednarodne lingvistične olimpijade v Bolgariji. 

Leta 2008 se je vpisal na študij fizike na Fakulteti za matematiko in fiziko v Ljubljani. Leta 2011 je zaključil študij prve bolonjske stopnje, leta 2013 pa še študij druge stopnje Računalniška fizika, s čimer si je pridobil naziv Magister fizike. Med študijem je trikrat v sklopu Googlovega programa Summer of Code razvijal odprtokodne programe. Za delo na programu za pridobivanje podatkov Orange je prejel državno Nagrado za prispevek k trajnostnem razvoju družbe, ki jo podeljujejo študentom z izjemnimi dosežki, za magistrsko delo pa fakultetno Prešernovo nagrado. 

Trenutno je Miha zaposlen kot mladi raziskovalec pri prof. Slobodanu Žumru na Fakulteti za matematiko in fiziko. V sklopu svojega dela razvija numerične simulacije širjenja svetlobe skozi zapletene tekočekristalne strukture in sklopitve med tekočim kristalom in svetlobo. 

\end{document}
