\documentclass[a4paper]{article}
\usepackage[utf8]{inputenc}
% \usepackage[slovene]{babel}

\title{Curriculum Vitae}
\author{Miha \v Can\v cula}
\date{March 2013}

\begin{document}
\pagenumbering{gobble}
\maketitle

I was born on April 10, 1989 in Postojna, Slovenia, and started elementary school in Ljubljana, Slovenia in 1996. While in elementary school, I took part in knowledge competitions from various fields and won first places at national competitions in physics, chemistry, logic and computer programming. 

After completing the elementary school in 2004, I enrolled into Gimnazija Bezigrad, a highly-regarded local high school. I was awarded the Slovenian national scholarship for gifted students, which I am still receiving now. While in high school, I focused mostly on mathematics and physics, where I participated in both national and international competitions. In the summer of 2007, I attended the International Mathematical Olympiad in Hanoi, Vietnam, where I received a bronze medal. The following year, in 2008, I attended the International Physical Olympiad, which also took place in Hanoi, and I again received a bronze medal. In the same year, I attended the International Linguistics Olympiad in Slanchev Bryag, Bulgaria. 

I graduated from high school in 2008 and started studying physics at the Faculty of mathematics and physics, University of Ljubljana. In 2011, I completed the first Bologna cycle and enrolled into the second-cycle program of Computational physics. I spent two summers writing free software as part of the Google Summer of Code programme. For my work on one of these projects, a data mining program called Orange, I received a national award for contribution to sustainable devepment of society, given to students with exceptional achievements. 

At the moment I am writing my master's thesis under the mentorship of prof. Slobodan Žumer. My work is about numerical simulations of propagation of light through liquid crystal structures. After completing the second cycle, I plan to continue my work with prof. Žumer as a PhD student, keeping my focus on computer simulations of liquid crystals. 

My mother tongue is Slovene, but I am also fluent in English. 

\end{document}
